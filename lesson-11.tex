\documentclass{article}
\usepackage[utf8]{inputenc}
\title{Lesson 11 - Discrete Mathematics}
\author{Matt Chung}
\date{November 9th, 2017}
\usepackage{tikz}
\usepackage{verbatim}
\usepackage{listings}
\usepackage{wasysym}
\usepackage{amsmath}
\usepackage{mathtools}
\renewcommand{\thesubsection}{\thesection.\alph{subsection}}

\begin{document}
\maketitle

\section{}

If $a > 0$ and $b > 0$ then $ab > 0$. Now prove the following:

\begin{itemize}
    \item If $a < 0$ and $b < 0$, then $ab > 0$ 
    \item If $a < 0$ and $b > 0$, then $ab < 0$
\end{itemize}

%Suppose $a < 0$ and $b < 0$. Using the ordering of integers rule number 2, we can rewrite the expression $a < b$ and $c<0$, then $bc < ac$; let's substitute $b$ for 0 and $c$ for $b$, giving us: $a < 0$ and $b < 0$, then $0b < a0$. Okay, this is obviously wrong because anything multiplied by zero gives us zero. So, zero CANNOT be less than (or equal to) zero. So, what the hell?

% This is me just taking notes. I'm scrambling right now, unsure where to begin. So what I'm doing, instead, is re-reading snippets of the chapter, trying to grasp how to formulate the proof.  Cause the question now is: which rule should I use? There are so many: associate, commutative. But, I'm guessing something with identity.

%Okay, still lost. I'm try a different approach. I'll just try formulating the answer without all the technical jargon.

%Going to try and explore the concept of ordering of integers being relation: transitive, antisymmetric, and reflexive. Maybe simmering on this idea will spark a new thought.

%Let's try rewriting the formula. Let's stick with $a < 0$ but switch things up for $b$: $0 > b$. Now that we flipped these two, I'm going to try and apply one of the rules:

\subsection{}

Since $a<0$, multiplying both sides of $b < 0$, will give us $ab > a0$ according to one of the rules of inequalities. A theorem in this lesson tells us $a0 = 0$. So $ab > a0$ is the same as $ab > 0$, and that is what we need to show. $\clubsuit$

\subsection{}

Since $a<0$, multiplying both sides of $b > 0$, will give us $ab < a0$, according to one of the rules of inequalities. A theorem in this lesson tells us $a0 = 0$. So $ab < a0$ is the same as $ab < 0$, and that is what we need to show. $\clubsuit$.

\section{}


If $ab=0$,then either $a=0$ or $b=0$.

We can use an indirect proof to prove the following:

\begin{itemize}
    \item a and b are both positive
    \item a is positive and b is negative
    \item a is negative and b is positive
    \item a is negative and b is negative
\end{itemize}

In the first case, both $a$ and $b$ are positive. In other words, $a >0$ and $b > 0$. If we multiply $a > 0$ by $b$, we get: $ab >0$ because of the rules of the inequalities in the lesson. Therefore, $ab$ is definitely not 0. 

In the second case, $a$ is positive and $b$ is negative. In other words, $a > 0$ and $b < 0$. If we multiply $a > 0$ by $b$, we get $ab < 0$ due to the rules of inequalities in the lesson. Therefore, $ab$ is not 0.

In the third case, $a$ is negative and $b$ is positive. In other words, $a < 0$ and $b > 0$. Similar to the first and second case, we multiply $a < 0$ by $b$, giving us $ab < 0$ due to the rules of inequalities in this lesson. Therefore, $ab$ is not 0.

Finally, in the third case, $a$ is negative and $b$ is negative. In other words, $a < 0$ and $b < 0$. When we multiply $a < 0$ by $b$, we get $ab > 0$ due to the rules of inequalities in this lesson. Therefore, $ab$ is not 0. So collectively, this proves that $ab=0$ if $a=0$ or $b=0$. $\clubsuit$.

\section{}

Determine all integers that 0 divides.

\textbf{Solution:} $\{0\}$

The question is asking us to find all integers $a$ which satisfy: $0|a$. This can be expressed as: $0c = a$. Since we know that $0$ multiplied by any integer results $0$, then we know that the only integer that satisfies this statement is $0$.

\section{}

For integers $r, s, t$, if $r|s$, then $r|st$.

%So, basically, we need a proof for this. So can I prove this—via a direct proof?

%What must I do, mathematically, to prove that $s$ multiplied to $t$ keeps the following true: $r|st$.

\textbf{Solution}:

Suppose $r, s$ and $r|s$. That means, $rc=s$ for some integer $c$. Let's multiply each side by $t$, giving us $rct = st$.  We can use the associative law to regroup the multiplication, giving us: $r(ct) = st$. This is equivalent to $r|st$, what we are trying to prove. $\clubsuit$.

\section{}

Determine the quotient and remainder when 107653 is divided by 22869.

\textbf{Solution:}

\begin{align*} 
a = qd + r \\
107653 = 22869(4) + 16177
\end{align*}

I basically plugged 107653 into the calculator, dividing the integer by 22869; the answer was 4.707 so I then multiplied 22869 by 4 (the quotient), subtracting that number from 107653, which then gave me the remainder: 16177.

\section{Bonus}

If $p$ is a prime, then $2p + 1$ is a prime.

This is absolutely false. Here is a counter example.

To disprove this theorem, we explicity specify an integer $p$ that is a prime but $2p + 1$ is not a prime.  In fact, 7 is a prime number, and if we substitute it in for $p$ then we get: $2(7) + 1 = 15$. This is not a prime since it is divisible by 5 and 3. So $p=7$ is a counterxample to this proposition. $\clubsuit$
\end{document}