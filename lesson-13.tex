\documentclass{article}
\usepackage[utf8]{inputenc}
\title{Lesson 13 - Discrete Mathematics}
\author{Matt Chung}
\date{December 7, 2017}
\usepackage{tikz}
\usepackage{verbatim}
\usepackage{listings}
\usepackage{wasysym}
\usepackage{amsmath}
\usepackage{mathtools}
\renewcommand{\thesubsection}{\thesection.\alph{subsection}}

\begin{document}
\maketitle

\section{}

\textbf{Solution:} $ 2^{2} \cdot 3^{1} \cdot 5^{2} \cdot 7^{1} \cdot 101^{1}$

I'm brute forcing this, starting with dividing 212100 by 100.  My manual steps look like this: $100 \cdot 2121 = 10 \cdot 10 \cdot 2121 = 10 \cdot 10 \cdot 21 \cdot 101 = 10 \cdot 10 \cdot 7 \cdot 3 \cdot 101 = 3^{1} \cdot 5^{2} \cdot 2^{2} \cdot 7 \cdot 101$

\section{}
\textbf{Solution: } 72 positive divisors.

Based off of the fundamental theorem, coupled with the prime factorization from the previous problem, we know the positive divisors of 212100 must look like $2^{a} \cdot 3^{b} \cdot 5^{c} \cdot 7^{d} \cdot 101^{e}$ where $a=0,1,2; b=0,1; c=0,1,2; d=0,1; e=0,1$. So, there are 72 positive divisors of 212100:\\

\noindent $2^{0}3^{0}5^{0}7^{0}101^{0}=1$\\
$2^{1}3^{0}5^{0}7^{0}101^{0}=2$\\
$2^{2}3^{0}5^{0}7^{0}101^{0}=4$\\
$2^{0}3^{1}5^{0}7^{0}101^{0}=3$\\
$2^{1}3^{1}5^{0}7^{0}101^{0}=6$\\
$2^{2}3^{1}5^{0}7^{0}101^{0}=12$\\
$2^{0}3^{0}5^{1}7^{0}101^{0}=5$\\
$2^{1}3^{0}5^{1}7^{0}101^{0}=10$\\
$2^{2}3^{0}5^{1}7^{0}101^{0}=20$\\
$2^{0}3^{1}5^{1}7^{0}101^{0}=15$\\
$2^{1}3^{1}5^{1}7^{0}101^{0}=30$\\
$2^{2}3^{1}5^{1}7^{0}101^{0}=60$\\
$2^{0}3^{0}5^{2}7^{0}101^{0}=25$\\
$2^{1}3^{0}5^{2}7^{0}101^{0}=50$\\
$2^{2}3^{0}5^{2}7^{0}101^{0}=100$\\
$2^{0}3^{1}5^{2}7^{0}101^{0}=75$\\
$2^{1}3^{1}5^{2}7^{0}101^{0}=150$\\
$2^{2}3^{1}5^{2}7^{0}101^{0}=300$\\
$2^{0}3^{0}5^{0}7^{1}101^{0}=7$\\
$2^{1}3^{0}5^{0}7^{1}101^{0}=14$\\
$2^{2}3^{0}5^{0}7^{1}101^{0}=28$\\
$2^{0}3^{1}5^{0}7^{1}101^{0}=21$\\
$2^{1}3^{1}5^{0}7^{1}101^{0}=42$\\
$2^{2}3^{1}5^{0}7^{1}101^{0}=84$\\
$2^{0}3^{0}5^{1}7^{1}101^{0}=35$\\
$2^{1}3^{0}5^{1}7^{1}101^{0}=70$\\
$2^{2}3^{0}5^{1}7^{1}101^{0}=140$\\
$2^{0}3^{1}5^{1}7^{1}101^{0}=105$\\
$2^{1}3^{1}5^{1}7^{1}101^{0}=210$\\
$2^{2}3^{1}5^{1}7^{1}101^{0}=420$\\
$2^{0}3^{0}5^{2}7^{1}101^{0}=175$\\
$2^{1}3^{0}5^{2}7^{1}101^{0}=350$\\
$2^{2}3^{0}5^{2}7^{1}101^{0}=700$\\
$2^{0}3^{1}5^{2}7^{1}101^{0}=525$\\
$2^{1}3^{1}5^{2}7^{1}101^{0}=1050$\\
$2^{2}3^{1}5^{2}7^{1}101^{0}=2100$\\
$2^{0}3^{0}5^{0}7^{0}101^{1}=101$\\
$2^{1}3^{0}5^{0}7^{0}101^{1}=202$\\
$2^{2}3^{0}5^{0}7^{0}101^{1}=404$\\
$2^{0}3^{1}5^{0}7^{0}101^{1}=303$\\
$2^{1}3^{1}5^{0}7^{0}101^{1}=606$\\
$2^{2}3^{1}5^{0}7^{0}101^{1}=1212$\\
$2^{0}3^{0}5^{1}7^{0}101^{1}=505$\\
$2^{1}3^{0}5^{1}7^{0}101^{1}=1010$\\
$2^{2}3^{0}5^{1}7^{0}101^{1}=2020$\\
$2^{0}3^{1}5^{1}7^{0}101^{1}=1515$\\
$2^{1}3^{1}5^{1}7^{0}101^{1}=3030$\\
$2^{2}3^{1}5^{1}7^{0}101^{1}=6060$\\
$2^{0}3^{0}5^{2}7^{0}101^{1}=2525$\\
$2^{1}3^{0}5^{2}7^{0}101^{1}=5050$\\
$2^{2}3^{0}5^{2}7^{0}101^{1}=10100$\\
$2^{0}3^{1}5^{2}7^{0}101^{1}=7575$\\
$2^{1}3^{1}5^{2}7^{0}101^{1}=15150$\\
$2^{2}3^{1}5^{2}7^{0}101^{1}=30300$\\
$2^{0}3^{0}5^{0}7^{1}101^{1}=707$\\
$2^{1}3^{0}5^{0}7^{1}101^{1}=1414$\\
$2^{2}3^{0}5^{0}7^{1}101^{1}=2828$\\
$2^{0}3^{1}5^{0}7^{1}101^{1}=2121$\\
$2^{1}3^{1}5^{0}7^{1}101^{1}=4242$\\
$2^{2}3^{1}5^{0}7^{1}101^{1}=8484$\\
$2^{0}3^{0}5^{1}7^{1}101^{1}=3535$\\
$2^{1}3^{0}5^{1}7^{1}101^{1}=7070$\\
$2^{2}3^{0}5^{1}7^{1}101^{1}=14140$\\
$2^{0}3^{1}5^{1}7^{1}101^{1}=10605$\\
$2^{1}3^{1}5^{1}7^{1}101^{1}=21210$\\
$2^{2}3^{1}5^{1}7^{1}101^{1}=42420$\\
$2^{0}3^{0}5^{2}7^{1}101^{1}=17675$\\
$2^{1}3^{0}5^{2}7^{1}101^{1}=35350$\\
$2^{2}3^{0}5^{2}7^{1}101^{1}=70700$\\
$2^{0}3^{1}5^{2}7^{1}101^{1}=53025$\\
$2^{1}3^{1}5^{2}7^{1}101^{1}=106050$\\
$2^{2}3^{1}5^{2}7^{1}101^{1}=212100$\\

\section{}
\textbf{Solution: } There are no integer solutions since the gcd, 7, does not divide into 69.

\begin{align*}
   14x + 77y &= 69 \\
   gcd(77, 14) &= (5)(14) + 7 \\
   gcd(14, 7) &= (2)(7) + 0 \\
   gcd(7, 0) &= 7 \\
\end{align*}

\section{}

\textbf{Solution:} $x = -50 + 11k; y = 10 -2k$ \\

\begin{align*}
   14x + 77y &= 70 \\
   gcd(77, 14) &= (5)(14) + 7 \\
   gcd(14, 7) &= (2)(7) + 0 \\
   gcd(7, 0) &= 7 \\
\end{align*}

Euclidean extended algorithm.

\begin{align*}
   &(77(1) + 14(0)) - (77(0) + 14(1)) * 5 = 77(1) + 14(-5) = 7 \\
   &(77(0) + 14(1)) - (77(1) + 14(-5)) * 2 = 77(-2) + 14(11) = 0 \\
\end{align*}

Based off of the Euclidean extended algorithm, we take the equation -- $ 77(1) + 14(-5) = 7$ -- and multiply both sides by 10, giving us: $77(10) + 14(-50) = 70$. 

Using the formulas that produce all solutions once one is known we get all solutions are
given by:\\

$x = -50 + (77/7)k = -50 + 11k$ and $y = 10 - (14/7)k = 10 - 2k$ where $k$ is any integer. For example, when $k = 5$, we get the solution: $x=5; y=0$. Putting this back into our equation, we get: $14(5) + 77(0) = 70$. Or, in another instance, when $k = 6$, we get the solution: $x=16; y=-2$. Similarly, we get: $14(16) + 77(-2) = 70$.


\section{}

\textbf{Solution: } $79(40) + 41(78) = 6358$. In other words, 40 video game machines and 78 video games. \\

$79x + 41y = 6358$

\begin{align*}
   gcd(79, 41) &= (1)(41) + 38 \\
   gcd(41, 38) &= (1)(38) + 3 \\
   gcd(38, 3) &= (12)(3) + 2 \\
   gcd(3, 2) &= (1)(2) + 1 \\
   gcd(2, 1) &= (2)(1) + 0 \\
   gcd(1, 0) &= 1
\end{align*}

\begin{align*}
   &(79(1) + 41(0)) - (79(0) + 41(1)) * 1 = 79(1) + 41(-1) = 38 \\
   &(79(0) + 41(1)) - (79(1) + 41(-1)) * 1 = 79(-1) + 41(2) = 3 \\
   &(79(1) + 41(-1)) - (79(-1) + 41(2)) * 12 = 79(1) + 41(-1) + 79(12) + 41(-24) = 79(13) + 41(-25) = 2 \\
\end{align*}

With the last linear combination, we can multiply the equation by 3179, giving us: $79(41327) + 41(-79475) = 6358$.

Using the formulas that produce all solutions once one is known we get all solutions are
given by: $x = 41327 + (41/1)k$ and $y = -79475 - (79/1)k$.

Although this linear combination satisfies the equation, it does not work since we cannot have negative number of video machines or negative number of video games. For example: $x = 41327 + 41 = 41368$ and $y = -79475 - 79 = -79554$. 

But since we cannot have negative numbers, we want something like:

$41327 + 41k >= 0$ and $-79475 - 79k >= 0$. Which means, $k >= \frac{-41327}{41}$ and $k \le \frac{79475}{-79}$. In other words, $\frac{-41327}{41} \le k \le \frac{79475}{-79}$. Therefore, the only number is where $k = -1007$. This gives us: $x = 41327 + (41)(-1007) = 40$ and $y= -79475 - (79)(-1007) = 78$. Put another way: $79(40) + 41(78) = 6358$.



\section{Bonus}

\textbf{Solution: } = $x = 297 + \frac{-7}{1}k$ and $y = 198 - \frac{5}{1}k$ \\

$5x - 7y = 99$

First, we need to find the gcd of 7 and 5. Although the gcd is obviously 1, here's the work:

\begin{align*}
   gcd(-7, 5) &= (-1)(5) - 2\\
   gcd(5, -2) &= (-2)(-2) + 1\\
   gcd(-2, 1) &= (2)(-1) + 0\\
   gcd(1, 0) &= 1
\end{align*}

Next, let's use the euclidean extended algorithm to find the linear combination who's result divides into 99.

\begin{align*}
   &(-7(1) + 5(0)) - (-7(0) + 5(1)) * -1 = -7(1) + 5(1) = -2\\
   &(-7(0) + 5(1)) - (-7(1) + 5(1)) * -2 = 5(1) + -7(2) + 5(2) = -7(2) + 5(3) = 1 \\
\end{align*}

$-7(2) + 5(3) =1 $. We take this equation and multiply both sides by 99, giving us: $-7(198) + 5(297) = 99$. Or, $5(297) - 7(198) = 99$.

Now, we can proceed with finding our general equation.

$x = 297 + \frac{-7}{1}k$ and $y = 198 - \frac{5}{1}k$

So, let's test this equation out, plugging in $k=2$.

$x = 297 + \frac{-7}{1}2 = 297 - 14 = 283$ and $y = 198 - \frac{5}{1}2 = 198 - 10 = 188$. With $x=283; y=188$, let's plug that into our original equation.

$5(283) - 7(188) = 99$






\end{document}