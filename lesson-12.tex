\documentclass{article}
\usepackage[utf8]{inputenc}
\title{Lesson 12 - Discrete Mathematics}
\author{Matt Chung}
\date{November 26, 2017}
\usepackage{tikz}
\usepackage{verbatim}
\usepackage{listings}
\usepackage{wasysym}
\usepackage{amsmath}
\usepackage{mathtools}
\renewcommand{\thesubsection}{\thesection.\alph{subsection}}

\begin{document}
\maketitle

\section{}

\subsection{}
\textbf{Solution: } 1

\begin{align*}
   gcd(233,89) = 233 &= (2)(89) + 55 \\
   gcd(89, 55) = 89 &= (1)(55) + 34 \\
   gcd(55, 34) = 55 &= (1)(34) + 21 \\
   gcd(34, 24) = 34 &= (1)(21) + 13 \\
   gcd(21, 13) = 21 &= (1)(13) + 8 \\
   gcd(13, 8) = 13 &= (1)(8) + 5 \\
   gcd(8, 5) = 8 &= (1)(5) + 3 \\
   gcd(5, 3) = 5 &= (1)(3) + 2 \\
   gcd(3, 2) = 3 &= (1)(2) + 1 \\
   gcd(2, 1) = 2 &= (2)(1) + 0 \\
   gcd(1, 0) = 1 
\end{align*}

\subsection{}
\textbf{Solution: } 13

\begin{align*}
   gcd(1001, 13) = 1001 &= (77)(13) + 0 \\
   gcd(13,0) = 13 \\
\end{align*}   

\subsection{}
\textbf{Solution: } 27

\begin{align*}
   gcd(2457, 1458) = 2457 &= (1)(1458) + 999 \\
   gcd(1458, 999) = 1458 &= (1)(999) + 459 \\
   gcd(999, 459) = 999 &= (2)(459) + 81 \\
   gcd(459, 81) = 459 &= (5)(81) + 54 \\
   gcd(81, 54) = 81 &= (1)(54) + 27 \\
   gcd(54, 27) = 54 &= (2)(27) + 0 \\
   gcd(27, 0) = 27
\end{align*}

\subsection{}

\begin{align*}
   gcd(567, 349) = 567 &= (1)(349) + 218 \\
   gcd(349, 218) = 349 &= (1)(218) + 131 \\
   gcd(218, 131) = 218 &= (1)(131) + 87 \\
   gcd(131, 87) = 131 &= (1)(87) + 44 \\
   gcd(87, 44) = 87 &= (1)(44) + 43 \\
   gcd(44, 43) = 43 &= (1)(43) + 1 \\
   gcd(1, 0) &= 1 
\end{align*}

\subsection{}
\textbf{Solution: } 9

\begin{align*}
   gcd(987654321, 123456789) = 987654321 &= (9)(123456789) + 9 \\
   gcd(123456789, 9) = 123456789 &= (13717421)(9) + 0 \\
   gcd(9, 0) &= 9
\end{align*}

\section{}

\textbf{Solution: } The GCD when either number is a prime is 1.

\begin{align*}
   gcd(7, 4) = 7 &= (1)(4) + 3 \\
   gcd(4, 3) = 4 &= (1)(3) + 1 \\
   gcd(3, 1) = 3 &= (3)(1) + 0 \\
   gcd(1, 0) &= 1 
\end{align*}

\begin{align*}
   gcd(11, 2) = 11 &= (5)(2) + 1 \\
   gcd(2, 1) = 2 &= (2)(1) + 0 \\
   gcd(1, 0) &= 1 
\end{align*}

\section{}

Wrapping my mind around the extended Euclidean algorithm was challenging and while solving this problem, I accidentally flipped the sign for one of the $s_{i}$, forcing me to spend about an hour tracing back to the root of the problem.

\begin{align*}
   gcd(6123 ,2913) = 6123 &= (2)(2913) + 297 \\
   gcd(2913,297) = 2913 &= (9)(297) + 240 \\
   gcd(297, 240) = 297 &= (1)(240) + 57 \\
   gcd(240,57) = 240 &= (4)(57) + 12 \\
   gcd(57, 12) = 57 &= (4)(12) + 9 \\
   gcd(12, 9) = 12 &= (1)(9) + 3 \\
   gcd(9, 3) = 9 &= (3)(3) + 0 \\
   gcd(3, 0) = 3
\end{align*}

\begin{center}
\begin{tabular}{ |c|c|c|c|c| } 
 \hline
 i & $s_{i}$ & $t_{i}$ & $r_{i}$ & $q_{i-1}$ \\ 
 -1 & 1 & 0 & 6123 & - \\ 
  0 & 0 & 1 & 2913 & - \\ 
  1 & 1 & -3 & 297 & 2 \\ 
  2 & -9 & 19 & 240 & 9 \\ 
  3 & 10 & -21 & 57 & 1 \\ 
  4 & -49 & 103 & 12 & 4 \\ 
  5 & 206 & -433 & 9 & 4 \\ 
  6 & -255 & 536 & 3 & 1 \\ 
  7 & 971 & -2041 & 0 & 3 \\ 
  \hline
\end{tabular}
\end{center}

\begin{align*}
   (6123(1) + 2913(0)) - (6123(0) + 2913(1) * 2 &= 6123(1) + 2913(-2) = 297 \\
   (6123(0) + 2913(1)) - (6123(1) + 2913(-2)) * 9  &= 2913(1) + 6123(-9) + 2913(18) \\
                                                   &= 6123(-9) + 2913(19) = 240 \\
   (6123(1) + 2913(-2) - (6123(-9) + 2913(19)) * 1 &= 6123(1) + 2913(-2) + 6123(9) + 2913(-19) \\  
                                                   &= 6123(10) + 2913(-21) = 57 \\
   (6123(-9) + 2913(19) - (6123(10) + 2913(-21)) * 4 &= 6123(-9) + 2913(19) + 6123(-40) + 2913(84) \\ 
                                                  &= 6123(-49) + 2913(103) = 12 \\
   (6123(10) + 2913(-21)) - (6123(-49) + 2913(103)) * 4 &= 6123(10) + 2913(-21) + 6123(196) + 2913(-412) \\ 
                                                  &= 6123(206) + 2913(-433) = 9 \\
   (6123(-49) + 2913(103)) - (6123(206) + 2913(-433)) * 1 &= 6123(-49) + 2913(103) + 6123(-206) + 2913(433) \\ 
                                                  &= 6123(-255) + 2913(536) = 3 \\
   (6123(206) + 2913(-433)) - (6123(-255) + 2913(536)) * 3 &= 6123(206) + 2913(-433) + 6123(765) + 2913(-1608) \\ 
                                                  &= 6123(971) + 2913(-2041) = 0 \\
\end{align*}

\section{}

Since we know that all linear combinations are multiples of the GCD, according to the theorem in the lesson, we can conclude that the GCD is a factor of 8.

\section{}

\textbf{Solution:} The smallest positive integer that can be written as a linear combination is 7, the gcd.  This can be solved using the Euclidean algorithm (although I only realized that after the fact that I unnecessarily used the extended Euclidean algorithm).

\begin{align*}
   gcd(2191, 1351) = 2191 &= (1)(1351) + 840 \\
   gcd(1351, 840) = 1351 &= (1)(840) + 511 \\
   gcd(840, 511) = 840 &= (1)(511) + 329 \\
   gcd(511, 329) = 511 &= (1)(329) + 182 \\
   gcd(329, 182) = 329 &= (1)(182) + 147 \\
   gcd(182, 147) = 182 &= (1)(147) + 35 \\
   gcd(147, 35) = 147 &= (4)(35) + 7 \\
   gcd(35, 7) = 35 &= (5)(7) + 0 \\
   gcd(7, 0) = 7 \\
\end{align*}

\begin{center}
\begin{tabular}{ |c|c|c|c|c| } 
 \hline
 i & $s_{i}$ & $t_{i}$ & $r_{i}$ & $q_{i-1}$ \\ 
 -1 & 1 & 0 & 2913 & - \\ 
  0 & 0 & 1 & 1351 & - \\ 
  1 & 1 & -1 & 840 & 1 \\ 
  2 & -1 & 2 & 511 & 1 \\ 
  3 & 2 & -3 & 329 & 1 \\ 
  4 & -3 & 5 & 182 & 1 \\ 
  5 & 5 & -8 & 147 & 1 \\ 
  6 & -8 & 13 & 35 & 1 \\ 
  7 & 37 & -60 & 7 & 4 \\ 
  8 & -193 & 313 & 0 & 5 \\ 
  \hline
\end{tabular}
\end{center}

\begin{align*}
 (2191(1) + 1351(0)) - (2191(0) + 1351(1)) * 1 &= 2191(1) + 1351(-1) = 840 \\
 (2191(0) + 1351(1)) - (2191(1) + 1351(-1)) * 1 &= 1351(1) + 2191(-1) + 1351(1) \\
                                                &= 2191(-1) + 1351(2) = 511 \\
 (2191(1) + 1351(-1)) - (2191(-1) + 1351(2)) * 1 &= 2191(1) + 1351(-1) + 2191(1) + 1351(-2) \\
                                                 &= 2191(2) + 1351(-3) = 329 \\
 (2191(-1) + 1351(2)) - (2191(2) + 1351(-3)) * 1 &= 2191(-1) + 1351(2) + 2191(-2) + 1351(3) \\
                                                 &= 2191(-3) + 1351(5) = 182 \\
 (2191(2) + 1351(-3)) - (2191(-3) + 1351(5)) * 1 &= 2191(2) + 1351(-3) + 2191(3) + 1351(-5) \\
                                                 &= 2191(5) + 1351(-8) = 147 \\
 (2191(-3) + 1351(5)) - (2191(5) + 1351(-8)) * 1 &= 2191(-3) + 1351(5) + 2191(-5) + 1351(8) \\
                                                 &= 2191(-8) + 1351(13) = 35 \\
 (2191(5) + 1351(-8)) - (2191(-8) + 1351(13)) * 4 &= 2191(5) + 1351(-8) + 2191(32) + 1351(-52) \\ 
                                                 &= 2191(37) + 1351(-60) = 7 \\
 (2191(-8) + 1351(13)) - (2191(37) + 1351(-60)) * 5 &= 2191(-8) + 1351(13) + 2191(-185) + 1351(300) \\ 
                                                    &= 2191(-193) + 1351(313) = 0 \\
\end{align*}

\newpage

\section{}
\textbf{Solution: 1}

If $n$ is a positive integer, then what is $gcd(n, n+1)$? One way to detect a pattern is by experimenting, plugging in numbers for $n$ an $n+1$.

\begin{align*}
    gcd(10, 11) &= 1 \\
    gcd(12, 13) &= 1 \\
    gcd(99, 100) &= 1 \\
    gcd(599, 600) &= 1 \\
\end{align*}

\end{document}