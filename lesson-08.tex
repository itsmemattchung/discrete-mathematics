\documentclass{article}
\usepackage[utf8]{inputenc}
\title{Lesson 08 - Discrete Mathematics}
\author{Matt Chung}
\date{September 24 2017}
\usepackage{tikz}
\usepackage{verbatim}
\renewcommand{\thesubsection}{\thesection.\alph{subsection}}

\begin{document}
\maketitle

\section{}
\textbf{Solution:} 399

Since the initial term is 3 and the common difference is 4, we know that the sequence increases; the second number is 7, the third 11, the fourth 15, and so on.  But instead of enumerating each number, one by one, we can formulate an equation: $3 + 4(n-1)$. This equation holds true: if we replace $n$ with $1$, we get: $3 + 4(1-1) = 3 + 4(0) = 3$; similarly, if we substitute $n$ for $2$, we get: $3 + 4(2-1) = 3 + 4(1) = 3 + 4 = 7$. Therefore, let's substitute $n$ with $100$, giving us: $3 + 4(100-1) = 3 + 4(99) = 3 + 396 = 397$.

Now, since the question asks for the 100th term, we can replace $n$ with $100$: $3 + 4(100-1) = 3 + 4(99) = 3 + 396 = 397$

\section{}

\textbf{Solution: } $$\sum_{k=-2}^{4} (2k + 3) = 35$$

\noindent Here's our equation, with the variables substituted:

\noindent $(2(-2) + 3) + (2(-1) + 3) + (2(0) + 3) + (2(1) + 3) + (2(2) + 3) + (2(3) + 3) + (2(4) + 3)$  = 
$(-4 + 3) + (-2 + 3) + (0 + 3) + (2 + 3) + (4 + 3) + (6 + 3) + (8 + 3)$  = $(-1) + (1) + (3) + (5) + (7) + (9) + (11) = 35$

\pagebreak

\section{}

\textbf{Solution: } $$\sum_{i=0}^{99} (-\frac{1}{2})^{i} = \frac{2}{3}(1 - (-\frac{1}{2})^{100})$$

$\frac{a-ar^{n}}{1 -r } = a(\frac{1-r^{n}}{1-r})$

$\frac{1-1(-\frac{1}{2})^{100}}{1 -(-\frac{1}{2}) } = \frac{1-(-\frac{1}{2})^{100}}{1 -(-\frac{1}{2}) } =\frac{1-(-\frac{1}{2})^{100}}{\frac{3}{2} } = \frac{2}{3}(1 - (-\frac{1}{2})^{100})$

\section{}

\subsection{}

\textbf{Solution:} $a_0 = 2, a_1 = 3, a_2 = 8, a_3 = 63, a_4 = 3968$

\begin{itemize}
    \item $a_0 = 2$
    \item $a_1 = a^2_{1-1} - 1 = a^2_{0} - 1 = 2^2 - 1 = 3$
    \item $a_2 = a^2_{2-1} - 1 = a^2_{1} - 1 = 3^2 - 1 = 8$
    \item $a_3 = a^2_{3-1} - 1 = a^2_{2} - 1 = 8^2 - 1 = 63$
    \item $a_4 = a^2_{4-1} - 1 = a^2_{3} - 1 = 63^2 - 1 = 3968$
\end{itemize}

\subsection{}

\textbf{Solution: } $u_1 = 1; u_2 = 5; u_3 = 19, u_4 = 65; u_5 = 211$ and the closed form formula is: $3^n - 2^n$;

Here's what given: $u_1 = 1; u_2 = 5; u_n = 5u_{n-1} - 6u_{n-2}$

Let's substitute the values for next three terms:

\begin{itemize}
    \item $5u_{n-1} - 6u_{n-2} = 5u_{3-1} - 6u_{3-2} = 5u_{2} - 6u_{1} = 5(5) - 6(1) = 25 - 6 = 19$
    \item $5u_{n-1} - 6u_{n-2} = 5u_{4-1} - 6u_{4-2} = 5u_{3} - 6u_{2} = 5(19) - 6(5) = 95 - 30 = 65$
    \item $5u_{n-1} - 6u_{n-2} = 5u_{5-1} - 6u_{5-2} = 5u_{4} - 6u_{3} = 5(65) - 6(19) = 325 - 114 = 211$
\end{itemize}

Thanks to the hint from the text, I was able to guess a closed form formula, through trial and error. I arrived at the following formula: $n^3 - n^2$.

\begin{itemize}
    \item $3^n - 2^n = 3^1 - 2^1 = 3 - 2 = 1$
    \item $3^n - 2^n = 3^2 - 2^2 = 9 - 2 = 5$
    \item $3^n - 2^n = 3^3 - 2^3 = 27 - 8 = 19$
    \item $3^n - 2^n = 3^4 - 2^4 = 81 - 16 = 65$
    \item $3^n - 2^n = 3^5 - 2^5 = 243 - 32 = 211$
\end{itemize}

As an aside, I had originally formulated the follow equation: $n^{3} - (n^{2} -1)$. Surprisingly, this equation generated the correct values for the first few terms. But, after the fourth term, the formula started falling apart.

\section{}

\textbf{Solution:} $a_1 = a$; for $n \ge 2$,  $a_n = ra_{n-1}$ 

Given the solution above, let's see how we could compute the first four terms.

\begin{itemize}
    \item $a_1 = ra_{1-1} = ra_{0}$
    \item $a_2 = ra_{2-1} = ra_{1}$
    \item $a_3 = ra_{3-1} = ra_{2}$
    \item $a_4 = ra_{4-1} = ra_{3}$
\end{itemize}


\section{}

\textbf{Solution: } $$\sum_{i=1}^{n} \frac{1}{2i - 1}$$

Since I almost made the grave mistake of using the $n$ as the index (despite the text's warning), let's make sure that if I substitute $n$ with 5, we mirror following: $\frac{1}{1} + \frac{1}{3} + \frac{1}{5} + \frac{1}{7} + \frac{1}{9}$.

With $n$ equal to 5:

$\frac{1}{(2(1) -1)} = \frac{1}{(2 - 1)} = \frac{1}{1}$ \\

$\frac{1}{(2(2) -1)} = \frac{1}{(4 - 1)} = \frac{1}{3}$\\

$\frac{1}{(2(3) -1)} = \frac{1}{(6 - 1)} = \frac{1}{5}$\\

$\frac{1}{(2(4) -1)} = \frac{1}{(8 - 1)} = \frac{1}{7}$\\

$\frac{1}{(2(5) -1)} = \frac{1}{(10 - 1)} = \frac{1}{9}$\\


\end{document}